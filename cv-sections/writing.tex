%----------------------------------------------------------------------------------------
%	SECTION TITLE
%----------------------------------------------------------------------------------------

\cvsection{Projecten}

%----------------------------------------------------------------------------------------
%	SECTION CONTENT
%----------------------------------------------------------------------------------------

\begin{cventries}

%------------------------------------------------

\cventry
{Ontwikkelaar} % Role
{Drankensysteem 152e FOS De Kangoeroes} % Title
{www.poef.thomasdewulf.be} % Location
{2017} % Date(s)
{ 
\begin{cvitems}
\item {Ontwikkelen van een app en .NET Core API \& website voor leiding en oud-leiding waarop aangeduid kan worden welke dranken ze drinken tijdens de scoutswerking. Ter vervanging van een papieren "poeflijst"}
\end{cvitems}
}

\cventry
{Ontwikkelaar} % Role
{Webapplicatie fiscaal attesten} % Title
{}
{2018} % Date(s)
{ 
\begin{cvitems}
\item {Webapplicatie waarmee ouders van 152e FOS De Kangoeroes een fiscaal attest konden genereren. Geschreven in .NET Core en gehost op Azure.}
\end{cvitems}
}

\cventry
{Ontwikkelaar} % Role
{Mobiele applicatie Saamdagen} % Title
{}
{2019} % Date(s)
{ 
\begin{cvitems}
\item {Ontwikkelen van een applicatie in Kotlin en Android Architecture Components voor het jaarlijkse startweekend voor leiding van FOS Open Scouting. De applicatie diende als vervanging voor een papieren infoboekje. Gebruikers konden info en een agenda raadplegen, zoeken voor welke sessie ze ingeschreven waren. Daarnaast was er ook een grondplan aanwezig in de applicatie.}
\end{cvitems}
}

\end{cventries}